% Preamble
% ---
\documentclass{article}
\author{David Johnson}
\title{\LaTeX{} Template}
\date{}

% Packages
% ---
\usepackage{amsmath} % Advanced math typesetting
\usepackage[utf8]{inputenc} % Unicode support (Umlauts etc.)
\usepackage[english]{babel} % Change hyphenation rules
\usepackage{hyperref} % Add a link to your document
\usepackage{graphicx} % Add pictures to your document
\usepackage{listings} % Source code formatting and highlighting
\usepackage{siunitx} % Required for alignment

\sisetup{
  round-mode          = places, % Rounds numbers
  round-precision     = 2, % to 2 places
}

\begin{document}

        \maketitle{}
        \tableofcontents{}

        \newpage
        \listoffigures
        \listoftables

        \newpage
        \section{Text goes here} % On top of document hierarchy; automatically numbered
        \subsection{}
        \subsubsection{}
        \paragraph{} % Paragraphs have no numbering
        \subparagraph{}

        \begin{figure}
                \caption{Dummy figure}
        \end{figure}

        \begin{table}
                \caption{Dummy table}
        \end{table}

        \begin{table}[h!]
                \caption{Table with aligned units.}
                \begin{center}
                        \label{tab:table1}
                        \begin{tabular}{l|S|r} % <-- Changed to S here.
                                \textbf{Value 1} & \textbf{Value 2} & \textbf{Value 3}\\
                                $\alpha$ & $\beta$ & $\gamma$ \\
                                \hline
                                1 & 1110.1 & a\\
                                2 & 10.1 & b\\
                                3 & 23.113231 & c\\
                        \end{tabular}
                \end{center}
        \end{table}

        \begin{align}
                abcde\\
                f(x) &= x^2\\
                f'(x) &= 2x\\
                F(x) &= \int f(2x)dx\\
                F(x) &= \frac{1}{3}x^3\\
                I &=    \begin{bmatrix}
                                1 0\\
                                0 1
                        \end{bmatrix}\\
                \left(\frac{1}{\sqrt{x}}\right)\\
                \cap\\
                \cup
        \end{align}

\end{document}

