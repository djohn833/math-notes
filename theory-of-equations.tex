% Preamble
% ---
\documentclass{article}
\author{David Johnson}
\title{\LaTeX{} Template}
\date{}

% Packages
% ---
\usepackage{array,booktabs}
\usepackage{amsmath} % Advanced math typesetting
\usepackage[utf8]{inputenc} % Unicode support (Umlauts etc.)
\usepackage[english]{babel} % Change hyphenation rules
\usepackage{hyperref} % Add a link to your document
\usepackage{graphicx} % Add pictures to your document
\usepackage{listings} % Source code formatting and highlighting
\usepackage{siunitx} % Required for alignment

\sisetup{
  round-mode          = places, % Rounds numbers
  round-precision     = 2, % to 2 places
}

\begin{document}

        \maketitle{}
        \tableofcontents{}

        \newpage
        \listoffigures
        \listoftables

        \newpage
        \section{Text goes here} % On top of document hierarchy; automatically numbered
        \subsection{}
        \subsubsection{}
        \paragraph{} % Paragraphs have no numbering
        \subparagraph{}

        \begin{figure}
                \caption{Dummy figure}
        \end{figure}

        \begin{table}
                \caption{Dummy table}
        \end{table}

        \begin{table}[h!]
                \caption{Table with aligned units.}
                \begin{center}
                        \label{tab:table1}
                        \begin{tabular}{l|S|r} % <-- Changed to S here.
                                \textbf{Value 1} & \textbf{Value 2} & \textbf{Value 3}\\
                                $\alpha$ & $\beta$ & $\gamma$ \\
                                \hline
                                1 & 1110.1 & a\\
                                2 & 10.1 & b\\
                                3 & 23.113231 & c\\
                        \end{tabular}
                \end{center}
        \end{table}

        \section{Detached Coefficients}

        \subsection{Multiplication}

        Multiply $x^2 - x + 1$ times $x^2 + x + 1$.
        \begin{gather*}
        \begin{array}{rrrrrrr}
                       &    &  1 & -1 & 1\\
                \times &    &  1 &  1 & 1\\
                \midrule
                  &    &  1 & -1 & 1\\
                  &  1 & -1 &  1 \\
                1 & -1 & 1\\
                \midrule
                1 &  0 &  1 &  0 & 1\\
        \end{array}\\
        \end{gather*}

        The product is $x^4 + x^2 + 1$.

        \subsection{Division}

        Divide $x^8 + x^7 + 3x^4 - 1$ by $x^4 - 3x^3 + 4x + 1$.

        \begin{gather*}
        \begin{array}{rrrrrrrrrrrrrr}
                1 &  1 &  0 &   0 &   3 &    0 &    0 &    0 &  -1 & \multicolumn{1}{|r}{1} & -3 &  0 &  4 &  1\\ \cline{10-14}
                1 & -3 &  0 &   4 &   1 &      &      &      &     & \multicolumn{1}{|r}{1} &  4 & 12 & 32 & 82\\ \cline{1-5}
                  &  4 &  0 &  -4 &   2 &    0\\
                  &  4 & 12 &   0 &  16 &    4\\ \cline{2-6}
                  &    & 12 &  -4 & -14 &   -4 &    0\\
                  &    & 12 & -36 &   0 &   48 &   12\\ \cline{3-7}
                  &    &    &  32 & -14 &  -52 &  -12 &    0\\
                  &    &    &  32 & -96 &    0 &  128 &   32\\ \cline{4-8}
                  &    &    &     &  82 &  -52 & -140 &  -32 &  -1\\
                  &    &    &     &  82 & -246 &    0 &  328 &  82\\ \cline{5-9}
                  &    &    &     &     &  194 & -140 & -360 & -83\\
        \end{array}
        \end{gather*}
        The quotient is $x^4 + 4x^3 + 12x^2 + 32x + 82$ and the remainder is $194x^3 - 140x^2 - 360x - 83$.
       
        \section{Remainder theorem}

        The remainder of $f(x)$ divided by $(x - c)$ is $f(c)$.

        Corrolary. $f(x)$ is divisible by $(x - c)$ iff $f(c) = 0$.

        \subsection{Synthetic Division}

        Example. Divide $3x^6 - 7x^5 + 5x^4 - x^2 - 6x - 8$ by $x + 2$.
        \begin{gather*}
        \begin{array}{rrrrrrrr}
                \left.-2\right) & 3 &  -7 &  5 &   0 &  -1 &   -6 &                       -8\\      
                                &   &  -6 & 26 & -62 & 124 & -246 &                      504\\
                \cline{2-8}
                                & 3 & -13 & 31 & -62 & 123 & -252 & \multicolumn{1}{|r}{496}\\
        \end{array}
        \end{gather*}

        The quotient is $3x^5 - 13x^4 + 31x^3 - 62x^2 + 123x - 252$ and the remainder is $496$.
        \begin{align}
                abcde\\
                f(x) &= x^2\\
                f'(x) &= 2x\\
                F(x) &= \int f(2x)dx\\
                F(x) &= \frac{1}{3}x^3\\
                I &=    \begin{bmatrix}
                                1 0\\
                                0 1
                        \end{bmatrix}\\
                \left(\frac{1}{\sqrt{x}}\right)\\
                \cap\\
                \cup
        \end{align}

\end{document}

