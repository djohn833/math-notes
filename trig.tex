% Preamble
% ---
\documentclass{article}
\author{David Johnson}
\title{\LaTeX{} Template}
\date{}

% Packages
% ---
\usepackage{amsmath} % Advanced math typesetting
\usepackage[utf8]{inputenc} % Unicode support (Umlauts etc.)
\usepackage[english]{babel} % Change hyphenation rules
\usepackage{hyperref} % Add a link to your document
\usepackage{graphicx} % Add pictures to your document
\usepackage{listings} % Source code formatting and highlighting
\usepackage{siunitx} % Required for alignment
\usepackage{tikz}

\sisetup{
  round-mode          = places, % Rounds numbers
  round-precision     = 2, % to 2 places
}

\begin{document}

        \maketitle{}
        \tableofcontents{}

        \newpage
        \listoffigures
        \listoftables

        \newpage
        \section{Text goes here} % On top of document hierarchy; automatically numbered
        \subsection{}
        \subsubsection{}
        \paragraph{} % Paragraphs have no numbering
        \subparagraph{}

        \begin{figure}
                \caption{Dummy figure}
        \end{figure}

        \begin{table}
                \caption{Dummy table}
        \end{table}

        \begin{table}[h!]
                \caption{Table with aligned units.}
                \begin{center}
                        \label{tab:table1}
                        \begin{tabular}{l|S|r} % <-- Changed to S here.
                                \textbf{Value 1} & \textbf{Value 2} & \textbf{Value 3}\\
                                $\alpha$ & $\beta$ & $\gamma$ \\
                                \hline
                                1 & 1110.1 & a\\
                                2 & 10.1 & b\\
                                3 & 23.113231 & c\\
                        \end{tabular}
                \end{center}
        \end{table}

        \begin{align}
                abcde\\
                f(x) &= x^2\\
                f'(x) &= 2x\\
                F(x) &= \int f(2x)dx\\
                F(x) &= \frac{1}{3}x^3\\
                I &=    \begin{bmatrix}
                                1 0\\
                                0 1
                        \end{bmatrix}\\
                \left(\frac{1}{\sqrt{x}}\right)\\
                \cap\\
                \cup
        \end{align}

        \section{Ratios}

        \begin{tikzpicture}
                \draw (0, 0) node[anchor=north] {$A$}
                   -- (4, 0) node[midway, below] {$b$} node[anchor=north] {$C$}
                   -- (4, 4) node[midway, right] {$a$} node[anchor=south] {$B$}
                   -- (0, 0) node[midway, above] {$h$};
        \end{tikzpicture}

        \begin{align}
                \sin \theta &= \frac{a}{h}\\
                \cos \theta &= \frac{b}{h}\\
                \tan \theta &= \frac{a}{b} = \frac{\sin \theta}{\cos \theta}\\
                \cot \theta &= \frac{b}{a} = \frac{1}{\tan \theta} = \frac{\cos \theta}{\sin \theta}\\
                \sec \theta &= \frac{h}{b} = \frac{1}{\cos \theta}\\
                \csc \theta &= \frac{h}{a} = \frac{1}{\sin \theta}\\
        \end{align}

        \section{Pythagorean Identities}

        \begin{align}
                \sin^2 \theta + \cos^2 \theta &= 1\\
                \tan^2 \theta + 1 &= \sec^2 \theta\\
                1 + \cot^2 \theta &= \csc^2 \theta
        \end{align}
                
        \section{Cofunction Identities}

        If $\alpha$ and $\beta$ are complementary ($\alpha + \beta = \ang{90} = \frac{\pi}{2}$), then
        \begin{align}
                \sin \alpha &= \cos \beta\\
                \tan \alpha &= \cot \beta\\
                \sec \alpha &= \csc \beta
        \end{align}

        \section{Law of Sines}

        If $\alpha$, $\beta$, and $\gamma$ are the angles of a triangle, $a$, $b$, and $c$ are the sides opposite them respectively, and $R$ is the radius of the circumcircle, then
        \[
                \frac{a}{\sin \alpha} = \frac{b}{\sin \beta} = \frac{c}{\sin \gamma} = 2R
        \]

        \section{Law of Cosines}

        If $\alpha$, $\beta$, and $\gamma$ are the angles of a triangle and $a$, $b$, and $c$ are the sides opposite them respectively, then

        \begin{align}
                a^2 &= b^2 + c^2 - 2bc \cos \alpha\\
                b^2 &= c^2 + a^2 - 2ac \cos \beta\\
                c^2 &= a^2 + b^2 - 2ab \cos \gamma
        \end{align}

        The last two formulas can be obtained by cyclic permutations of sides $a$, $b$, and $c$ and angles $\alpha$, $\beta$, and $\gamma$.

        \section{Triangle Solving}

        TODO

        \section{Even and Odd functions}
        
        A function is \textbf{even} iff, for all $x$, $f(-x) = f(x)$.\\
        A function is \textbf{odd} iff, for all $x$, $f(-x) = -f(x)$.\\
        $\cos$ and $\sec$ are even functions.\\
        $\sin$, $\tan$, $\cot$, and $\csc$ are odd functions.\\
        If $f(x)$ is any function, then \[ g(x) = \frac{1}{2}(f(x) + f(-x)) \] is an even function and \[ h(x) = \frac{1}{2}(f(x) - f(-x)) \] is an odd function.
        
        \begin{align}
                e^{i \theta} &= \underbrace{\cos \theta}_\text{even} + i \underbrace{\sin \theta}_\text{odd}\\
                e^{x} &= \underbrace{\cosh x}_\text{even} + \underbrace{\sinh x}_\text{odd}
        \end{align}

        \section{Sum and Difference of Angles}

        Note that the sum formulas must be symmetric in $\alpha$ and $\beta$ because addition is commutative. The difference formula for $\cos$ is also symmetric because $\cos$ is even, so \[ \cos(\alpha - \beta) = \cos(-(\alpha - \beta)) = \cos(\beta - \alpha) \]

        \begin{align}
                \sin(\alpha \pm \beta) &= \sin \alpha \cos \beta \pm \cos \alpha \sin \beta\\
                \cos(\alpha \pm \beta) &= \cos \alpha \cos \beta \mp \sin \alpha \sin \beta\\
                \tan(\alpha \pm \beta) &= \frac{\tan \alpha \pm \tan \beta}{1 \mp \tan \alpha \tan \beta}
        \end{align}

        \section{Double Angle}

        \begin{align}
                \sin 2 \alpha &= 2 \sin \alpha \cos \alpha\\
                \cos 2 \alpha &= \cos^2 \alpha - \sin^2 \alpha = 2 \cos^2 \alpha - 1 = 1 - 2 \sin^2 \alpha\\
                \tan 2 \alpha &= \frac{2 \tan \alpha}{1 - \tan^2 \alpha}
        \end{align}

        \section{Half Angle}

        \begin{align}
                \sin \frac{\alpha}{2} &= \pm \sqrt{\frac{1 - \cos \alpha}{2}}\\
                \cos \frac{\alpha}{2} &= \pm \sqrt{\frac{1 + \cos \alpha}{2}}\\
                \tan \frac{\alpha}{2} &= \frac{\sin \alpha}{1 + \cos \alpha} = \frac{1 - \cos \alpha}{\sin \alpha}
        \end{align}

        \section{Products to Sums}

        \begin{align}
                \cos \alpha \cos \beta &= \frac{1}{2} [ \cos(\alpha + \beta) + \cos(\alpha - \beta) ]\\
                \sin \alpha \sin \beta &= \frac{1}{2} [ \cos(\alpha - \beta) - \cos(\alpha + \beta) ]
        \end{align}
        
        \section{Sums to Products}

        \begin{align}
                \sin \alpha + \sin \beta &= 2 \sin \frac{\alpha + \beta}{2} \cos \frac{\alpha - \beta}{2}\\
                \sin \alpha - \sin \beta &= 2 \cos \frac{\alpha + \beta}{2} \sin \frac{\alpha - \beta}{2}\\
                \cos \alpha + \cos \beta &= 2 \cos \frac{\alpha + \beta}{2} \cos \frac{\alpha - \beta}{2}\\
                                         &= 2 \cos \frac{\alpha + \beta}{2} \cos \frac{\beta - \alpha}{2}\\
                \cos \alpha - \cos \beta &= -2 \sin \frac{\alpha + \beta}{2} \sin \frac{\alpha - \beta}{2}\\
                                         &= 2 \sin \frac{\alpha + \beta}{2} \sin \frac{\beta - \alpha}{2}
        \end{align}

        \section{Uniformization in terms of $\tan \frac{\alpha}{2}$}

        Let $t = \tan \frac{\alpha}{2}$. Then

        \begin{align}
                \sin \alpha &= \frac{2t}{1 + t^2}\\
                \cos \alpha &= \frac{1 - t^2}{1 + t^2}\\
                \tan \alpha &= \frac{2t}{1 - t^2}\\
                \cot \alpha &= \frac{1 + t^2}{2t}\\
                \sec \alpha &= \frac{1 + t^2}{1 - t^2}\\
                \csc \alpha &= \frac{1 + t^2}{2t}
        \end{align}

\end{document}

