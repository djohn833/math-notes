% Preamble
% ---
\documentclass{article}
\author{David Johnson}
\title{Lattice Theory}
\date{}

% Packages
% ---
\usepackage{amsmath} % Advanced math typesetting
\usepackage{enumitem}
\usepackage{multirow}

\begin{document}

        \maketitle{}
        \tableofcontents{}

        \newpage
        \section{Binary Relation Properties}
        For $a, b, c \in S$ and $O \subset S \cross S$...
        \begin{table}[h!]
                  \label{tab:table1}
                  \begin{tabular}{l|l|l}
                        \textbf{Group} & \textbf{Name} & \textbf{Expression}\\
                        \hline
                        \multirow{1}{*}{Reflexive} & Reflexive      & $aOa$\\
                        \hline
                        \multirow{2}{*}{Symmetric} & Symmetric      & $aOb \leftrightarrow bOa$\\
                                                   & Anti-symmetric & $aOb \wedge bOa \rightarrow a = b$\\
                        \hline
                        \multirow{1}{*}{Transitive} & Transitive    & $aOb \wedge bOc \rightarrow aOc$\\
                  \end{tabular}
        \end{table}

        \newpage
        \section{Definition of a Lattice}
        \subsection{Partial Order}
        A partial order is a binary relation that is reflexive, anti-symmetric, and transitive.
        \begin{align}
                abcde\\
                f(x) &= x^2\\
                f'(x) &= 2x\\
                F(x) &= \int f(2x)dx\\
                F(x) &= \frac{1}{3}x^3\\
                I &=    \begin{bmatrix}
                                1 0\\
                                0 1
                        \end{bmatrix}\\
                \left(\frac{1}{\sqrt{x}}\right)\\
                \cap\\
                \cup
        \end{align}

\end{document}

