% Preamble
% ---
\documentclass{article}
\author{David Johnson}
\title{Triangle of Power}
\date{}

% Packages
% ---
\usepackage{amsmath} % Advanced math typesetting
\usepackage[utf8]{inputenc} % Unicode support (Umlauts etc.)
\usepackage[english]{babel} % Change hyphenation rules
\usepackage{hyperref} % Add a link to your document
\usepackage{graphicx} % Add pictures to your document
\usepackage{listings} % Source code formatting and highlighting
\usepackage{siunitx} % Required for alignment

\usepackage{tikz}
\usetikzlibrary{shapes.geometric,calc,through}

\sisetup{
  round-mode          = places, % Rounds numbers
  round-precision     = 2, % to 2 places
}

\begin{document}
        \newcommand*{\equilateraltriangle}[1]{\node (#1) [regular polygon, regular polygon sides=3, draw, minimum width=40mm] at (0, 0){}}

        \maketitle{}
        \tableofcontents{}

        \newpage

        \begin{tikzpicture}
                \equilateraltriangle{tri};
                \node at (tri.corner 1) [anchor=south]{$3$};
                \node at (tri.corner 2) [anchor=north]{$2$};
                \node at (tri.corner 3) [anchor=north]{$8$};
        \end{tikzpicture}

        \begin{align*}
                2^3 &= 8\\
                \sqrt[3]{8} &= 2\\
                \log_{2} 8 &= 3
        \end{align*}

        \begin{tikzpicture}
                \equilateraltriangle{tri};
                \node at (tri.corner 1) [anchor=south]{exponent};
                \node at (tri.corner 2) [anchor=north]{base};
                \node at (tri.corner 3) [anchor=north]{power};
        \end{tikzpicture}
       
        $a$ and $b$ are dummies for bases, $x$ and $y$ are dummies for exponents, and $u$ and $v$ are dummies for the powers (the base raised to the exponent).

        \begin{tikzpicture}
                \equilateraltriangle{tri};
                \node at (tri.corner 1) [anchor=south]{$+$};
                \node at (tri.corner 2) [anchor=north]{constant base};
                \node at (tri.corner 3) [anchor=north]{$\times$};
        \end{tikzpicture}

        \begin{align*}
                a^{x}a^{y} &= a^{x+y}\\
                \log_a uv &= \log_a u + \log_a v
        \end{align*}

        \begin{tikzpicture}
                \equilateraltriangle{tri};
                \node at (tri.corner 1) [anchor=south]{constant exponent};
                \node at (tri.corner 2) [anchor=north]{$\times$};
                \node at (tri.corner 3) [anchor=north]{$\times$};
        \end{tikzpicture}

        \begin{align*}
                a^{x}b^{x} &= (ab)^{x}\\
                \sqrt[x]{u}\sqrt[x]{v} &= \sqrt[x]{uv}
        \end{align*}

        \begin{tikzpicture}
                \equilateraltriangle{tri};
                \node at (tri.corner 1) [anchor=south]{$\oplus$};
                \node at (tri.corner 2) [anchor=north]{$\times$};
                \node at (tri.corner 3) [anchor=north]{constant power};
        \end{tikzpicture}
        
        Here, $a \oplus b$ stands for the harmonic mean, $a \oplus b = \frac{1}{\frac{1}{a} + \frac{1}{b}}$..

        \begin{align*}
                \sqrt[x]{u}\sqrt[y]{u} &= \sqrt[x \oplus y]{u}\\
                \log_a u \oplus \log_b u &= \log_{ab} u
        \end{align*}
\end{document}

